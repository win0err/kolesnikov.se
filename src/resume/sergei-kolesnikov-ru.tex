% -*- indent-tabs-mode: t; tab-width: 4 -*-
\documentclass[10pt]{article}

% -*- indent-tabs-mode: t; tab-width: 4 -*-
\usepackage[a4paper, hmargin=20mm, vmargin=20mm, top=20mm]{geometry}


%% Language settings
\usepackage[utf8]{inputenc}
\usepackage[T2A]{fontenc}
\usepackage{polyglossia}
\setotherlanguages{english,british,russian}


%% Show dates
\usepackage[useregional]{datetime2}
\DTMlangsetup[en-GB]{ord=raise}  % ^{th}


%% File modification date
\usepackage{filemod}
\renewcommand*{\thefilemoddate}[3]{\DTMdisplaydate{#1}{#2}{#3}{-1}}


%% Fonts
\usepackage{fontspec}
\setmonofont{LiberationMono}[Scale=0.87]
\setmainfont{LiberationSerif}
\setsansfont{LiberationSans}


%% Conditional commands
\usepackage{ifthen, xifthen}  % \ifthenelse, \isempty tests


%% Colored links
\usepackage[dvipsnames]{xcolor}
\definecolor{darkblue}{HTML}{00008B}

\usepackage{hyperref}
\hypersetup{colorlinks=true, linkcolor=black, urlcolor=darkblue}
\urlstyle{same}


%% Graphics
\usepackage{graphicx} % Incude images
\usepackage{floatflt} % Floating positioning


%% Horizontal lists
\usepackage[inline]{enumitem}


%% Settings
\setcounter{secnumdepth}{0} % Suppress section numbering
\setlength\parindent{0pt} % Stop paragraph indentation


% ---------


%% Title
\renewcommand{\title}[1]{ {\huge{\textbf{#1}}}}


%% Horizontal list separated by bullet
\newenvironment{horizontalitemize}
{ \begin{itemize*}[
	label={},
	afterlabel={},
	itemjoin={{\enskip\textbullet{}\enskip}}
] }
{ \end{itemize*} }


%% Compact unordered list
\newenvironment{compactitemize}
{ \begin{itemize}[parsep=0pt,label={---}] }
{ \end{itemize} }


%% Super compact unordered list
\newenvironment{supercompactitemize}
{ \begin{itemize}[
  nosep,
  after=\vspace{.5\baselineskip},
  before=\vspace{.25\baselineskip},
  itemsep=.15em,
  parsep=0pt,
  label={---}
] }
{ \end{itemize} }


%% Job entry
\newcommand{\job}[7]{
	\filbreak % no page break

	\begin{tabbing}
		\hspace{2cm}  \= \kill
		\textbf{#1--} \> \href{#4}{#3}, #5 \\  % from-- Link To Employer, Office Location
		\textbf{#2}	  \>\+ \textit{#6}	   \\  % to		Position

					  \begin{minipage}{\the\dimexpr\textwidth-2cm\relax}
						\vspace{2mm}
					  	#7 % Job Description
					  \end{minipage}
	\end{tabbing}

	\vspace{-.25\baselineskip}
}


%% Education entry
\newcommand{\education}[6]{
	\begin{tabbing}
		\hspace{2cm} \= \kill
		\bf{#1--#2}	\> \href{#4}{#3}\ifthenelse{\not\isempty{#5}}{, #5}{} \\ % from--to Link To University[, Faculty]
					\> #6 % Degree
	\end{tabbing}
}


%% Table-like summary
% environment
\newenvironment{summary}
{
	\begin{tabbing}
	\hspace{3cm} \= \kill
}
{ \end{tabbing} }

% command
\newcommand{\summaryitem}[2]{
	\textit{#1:} \> #2\\
}


%% Bottom information about last update
\newcommand{\bottominfo}[1]{
	\par\vfill{\footnotesize\color{gray}#1}
}

\setmainlanguage{russian}

\begin{document}

\begin{floatingfigure}[r]{3cm}
	\vspace{-.5\baselineskip}
	\includegraphics[width=3cm]{userpic}
\end{floatingfigure}

\title{Сергей Колесников}
\vspace{-.5\baselineskip}

\begin{horizontalitemize}
	\item \href{mailto:sergei@kolesnikov.se}{sergei@kolesnikov.se}
	\item \href{https://t.me/win0err}{@win0err}
	\item \href{https://kolesnikov.se/contacts.html}{kolesnikov.se/contacts.html}
\end{horizontalitemize}

\begin{summary}
	\summaryitem{Дата рождения}{\DTMdisplaydate{1994}{11}{8}{-1}}
	\summaryitem{Языки}{Русский, Английский \textit{(B1)}}
\end{summary}

% ---------

\vspace{-2\baselineskip}
\section{О себе}

Я инженер-программист из Москвы, я разрабатываю веб-приложения и сервисы.
Я начинал свою карьеру бекенд-разработчиком, но~из-за любви к созданию интерфейсов я сфокусировался на разработке фронтенда.
Сторонник философии UNIX: предпочитаю создавать простые вещи и добавлять сложность только там, где это необходимо.
Я быстро учусь, внимателен к деталям, а также люблю компьютерные науки.

% ---------

\section{Опыт работы}

\job
	{Окт. 2022}{н.в.}
	{AnyCut}
	{https://any-cut.pro/}
	{Россия, Москва}
	{Senior Software Engineer}
	{
	  Запуск сервиса расчёта стоимости лазерной резки металла.

	  \begin{supercompactitemize}
	  	\item Сбор и анализ бизнес-требований
		\item Проектирование REST API, участие в разработке архитектуры сервиса и код-ревью
		\item Проектирование и~разработка клиентского SPA, встраиваемого виджета, \\
		  а также библиотеки UI-компонентов
		\item Написание юнит и end-to-end тестов
		\item Проектирование UI
	  \end{supercompactitemize}

	  \textit{Технологии:} JavaScript/TypeScript, Vue 3, Pinia, Playwright, i18n
	}

\job
	{Февр. 2019}{Окт. 2022}
	{Ingram Micro Cloud (CloudBlue)}
	{https://www.cloudblue.com/}
	{Россия, Москва}
	{Software Engineer $\rightarrow$ Senior Software Engineer}
	{
	  Дизайн, планирование и разработка фич для продукта \href{https://connect.cloudblue.com/community/}{CloudBlue Connect} --- SaaS-платформы для~дистрибуции облачных сервисов.

	  \begin{supercompactitemize}
	  	\item Проектирование и разработка фронтенд-частей сервисов уведомлений, брендинга и валидации, а также участие в разработке других частей продукта
		\item Разработка библиотеки UI-компонентов
		\item Написание юнит и E2E тестов, внедрение автоматического тестирования email-шаблонов
		\item Код-ревью
		\item Онбординг фронтенд-разработчиков и написание гайдлайнов
	  \end{supercompactitemize}

	  \textbf{Прочее}

	  \begin{supercompactitemize}
	  	\item Выступил на митапе <<MSK Vue.js>> с докладом <<\href{https://www.youtube.com/watch?v=Rz_RynHNZKg}{Функциональное программирование c Vue и Vuex. Сильные стороны. Наиболее эффективные точки применения}>>
		\item Написал статью <<\href{https://habr.com/ru/company/odin_ingram_micro/blog/526094/}{Организация типовых модулей во Vuex}>> в корпоративный блог на Хабре
	  \end{supercompactitemize}

	  \textit{Технологии:} JavaScript, React, Vue 2, Vuex, WebSocket, Cypress
	}

\job
	{Нояб. 2017}{Февр. 2019}
	{Тендертех}
	{https://tendertech.ru/}
	{Россия, Москва}
	{Software Engineer}
	{
	  Проектирование и разработка внутренней CRM-системы сервиса по выдаче банковских гарантий.

	  \begin{supercompactitemize}
	  	\item \textit{Бекенд:} проектирование и разработка CRM-системы, интеграция с почтовыми сервисами, создание алгоритма распределения заявок между агентами
		\item \textit{Фронтенд:} разработка веб-интерфейса системы и перевод её на React + Redux
		\item Код-ревью бекенда и фронтенда, внедрение статических анализаторов кода
	  \end{supercompactitemize}

	  \textit{Технологии:} Go, PHP, Slim, MySQL, PostgreSQL; JavaScript, React, Redux
	}

% ---------

\section{Преподавание и наставничество}

\job
	{Февр. 2021}{Июнь 2021}
	{МИРЭА --- Российский технологический университет}
	{https://mirea.ru/}
	{Россия, Москва}
	{Педагогическая практика}
	{
	  Проведение лабораторных занятий со студентами по предмету <<\textit{Технологии разработки программных приложений}>>: обучение работы с Docker, Docker Compose и Ansible.
	}

\job
	{Сент. 2019}{Июнь 2020}
	{МИРЭА --- Российский технологический университет}
	{https://mirea.ru/}
	{Россия, Москва}
	{Ассистент}
	{
	  Проведение практических и лабораторных занятий со студентами, а также составление материалов и заданий для лабораторных работ и экзаменов.

	  \begin{supercompactitemize}
	  	\item \textit{Объектно-ориентированное программирование.} Обучение основам Java и ООП.
		\item \textit{Технологии разработки программного обеспечения.} Обучение коллективной разработке SPA на~JavaScript + React, сборке приложений с помощью Webpack, юнит-тестированию, а также внедрению практик CI/CD на примере интеграции GitHub Actions (тестирование и~развертывание готового приложения в Heroku).
	  \end{supercompactitemize}
	}

% ---------

\section{Образование}

\education
	{2019}{2023}
	{МИРЭА --- Российский технологический университет}
	{https://mirea.ru/}
	{Информатика и вычислительная техника}
	{Исследователь. Преподаватель-исследователь}

\education
	{2017}{2019}
	{МИРЭА --- Российский технологический университет}
	{https://mirea.ru/}
	{Программная инженерия}
	{Магистр, \textit{с отличием}}

\education
	{2013}{2017}
	{МИРЭА --- Российский технологический университет}
	{https://mirea.ru/}
	{Программная инженерия}
	{Бакалавр}

% ---------

\section{Проекты}

\begin{compactitemize}
	\item \textbf{\href{https://kolesnikov.se/}{kolesnikov.se}}. Персональный веб-сайт с блогом и фотографиями, стилизованный под веб-дизайн 90-х
	\item \textbf{\href{https://github.com/cloudblue/material-svg}{Material SVG}}. Автоматически обновляемый набор иконок Material Design, доступный в виде NPM-пакета
	\item \textbf{\href{https://github.com/win0err/gnome-runcat}{RunCat}}. Расширение для GNOME Shell, показывающее загрузку процессора скоростью бега котика
	\item \textbf{\href{https://github.com/win0err/twtwt}{twtwt}}. Клиент для twtxt --- децентрализованного, минималистичного сервиса микроблогинга для хакеров
	\item \textbf{Velobike Statistics}. Проект по сбору статистики использования московского городского велопроката
\end{compactitemize}

Внес вклад в различные проекты с открытым исходным кодом, такие как:
\href{https://github.com/browserslist/browserslist/pulls?q=author:win0err}{Browserslist},
\href{https://github.com/ai/nanospy/pulls?q=author:win0err}{Nano Spy},
\href{https://github.com/jshmrtn/vue3-gettext/pulls?q=author:win0err}{Vue 3 Gettext},
\href{https://github.com/buckket/twtxt/pulls?q=author:win0err}{twtxt},
\href{https://github.com/tfeldmann/organize/pulls?q=author:win0err}{organize} и \href{https://github.com/lstrojny/functional-php/pulls?q=author:win0err}{Functional PHP}.

% ---------

\section{Профессиональные навыки}

\begin{compactitemize}
	\item \textbf{Языки программирования:} JavaScript/TypeScript --- основные языки, также пишу на Go, Python, PHP и C
	\item \textbf{Веб-разработка:} HTML, CSS (SASS, PostCSS), фреймворки Vue и React
	\item \textbf{СУБД:} MySQL/MariaDB, PostgreSQL, SQLite, Redis
	\item \textbf{Инструменты:} UNIX-подобные ОС (Linux/BDS/macOS), Git, \LaTeX, make, Vite, Webpack, Podman/Docker
\end{compactitemize}

% ---------

\section{Хобби}

Программирование, фотография, гаджеты и путешествия.

% ---------

%% \section{Рекомендации}

%% Доступны по запросу.

% ---------

\bottominfo{
	Дата обновления: \filemodprintdate{\jobname}\space
	Актуальная версия этого резюме доступна по адресу \url{https://kolesnikov.se/resume/\jobname.pdf}
}

\end{document}
